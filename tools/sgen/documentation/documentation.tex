\documentclass[a4paper,11pt]{article}
\usepackage[english]{babel}
\usepackage[utf8]{inputenc}
\usepackage{textcomp}
\usepackage[T1]{fontenc}
\addtolength{\oddsidemargin}{-.5in}
\addtolength{\evensidemargin}{-.5in}
\addtolength{\textwidth}{1in}

%for images
\usepackage[pdftex]{graphicx}

%control line spacing
\usepackage{setspace}
\singlespacing

\usepackage{fancyhdr}
\pagestyle{fancy}
\fancyhead[L]{\texttt{sgen} documentation} 
\fancyhead[R]{INRIA/LIP6 \today}

\usepackage{hyperref}
\hypersetup{pdfborder = {0 0 0}} %remove link borders

\title{\textbf{Documentation for \texttt{sgen}}}
\date{\today}
\author{Chi Pham}

\begin{document}
\maketitle
\tableofcontents
\newpage

\section{About}
We first provide information about the tool, its motivation, and its uses.
\subsection{What is \texttt{sgen}?}
\texttt{sgen} is a \texttt{Coccinelle}\footnote{\hyperref[http://coccinelle.lip6.fr/]{http://coccinelle.lip6.fr/}} metaprogramming tool that can generate hardened semantic patches for use in e.g. the Linux kernel.\\
Or, in less fancy words, \texttt{sgen} can take your simple \texttt{Coccinelle} script containing one or more rules with \texttt{*}, \texttt{+}, or \texttt{-}, and then output the same script with more options.\\\\
In particular, \texttt{sgen} generates the \texttt{patch}, \texttt{context}, \texttt{org}, and \texttt{report} virtual rules. These options are prevalent in the \texttt{Coccinelle} scripts included in the Linux kernel and are used in the following cases:
\begin{itemize}
\item \texttt{patch}: Used for +/- rules that transform the matched \texttt{C} code and outputs the changes in Unix \texttt{diff} format.
\item \texttt{context}: Used for * rules that find the matched \texttt{C} code and outputs it in \texttt{diff}-like format.
\item \texttt{org}: Used for script rules that output matches in \texttt{emacs} org mode format\footnote{\hyperref[http://orgmode.org/]{http://orgmode.org/}} with error message and line numbers.
\item \texttt{report}: Used for script rules that output matches with error message and line numbers.
\end{itemize}
\bigskip

\subsection{Features}
\texttt{sgen} includes (but is not limited to) support for
\begin{itemize}
\item Generating a \texttt{context} (aka *) version of a \texttt{patch} (aka +/-) rule.
\item Generating \texttt{org} and \texttt{report} (aka script) versions of both \texttt{patch} and \texttt{context} rules.
\item Adding \texttt{patch}, \texttt{context}, \texttt{org}, and \texttt{report} dependencies to rule headers.
\item Allowing the user to specify preface information for the generated rule, such as keywords, options, etc. as well as error messages for rules and names for nameless rules.
\item Automatic rulename and error message generation when none specified by the user.
\item Rule splitting to ensure correct \texttt{context} mode output for rules containing pattern matching disjunctions.
\item And more ...
\end{itemize}

\clearpage

\section{Usage}
This section contains information about how the tool is installed and used.
\subsection{Installation and uninstallation}
\textbf{To install}: You need to have \texttt{Coccinelle} and all its dependencies installed. You also need to have the \texttt{Coccinelle} source code available. By default, the \texttt{sgen} directory lies within the \texttt{Coccinelle} directory in \texttt{coccinelle/tools/sgen}.\footnote{If you change this, you need to modify the \texttt{COCCIDIR} path in \texttt{source/Makefile} to point to the \texttt{Coccinelle} source code directory.} From the \texttt{sgen} directory, do:
\begin{enumerate}
\item Run \texttt{make} to compile the program.
\item Run \texttt{make install}\footnote{If permission is denied, run it in superuser mode e.g. \texttt{sudo make install}, depending on your system.\label{sudo}} to install the program.
\item (Optional) Try out the program by running \texttt{sgen examples/tiny.cocci} or \texttt{sgen <path\_to\_your\_script>.cocci}.
\end{enumerate}
\textbf{To uninstall}: From the \texttt{sgen} directory, do:
\begin{enumerate}
\item Run \texttt{make uninstall}$^{\ref{sudo}}$. 
\end{enumerate}
\bigskip

\subsection{Running the program}
The most common usages are as follows, for a semantic patch file \texttt{foo.cocci} and an sgen config file \texttt{foo.config}:
\begin{itemize}
\item \texttt{sgen foo.cocci}: Generate the file with the information found
d in \texttt{foo.config} if it exists. If not, the program is run in interactive mode.
\item \texttt{sgen -{}-config foo.config foo.cocci}: Generate the file with
  \texttt{foo.config} as the configuration file. The shorthand \texttt{-c} can be used instead of \texttt{-{}-config}.
\item \texttt{sgen foo.cocci -{}-interactive}: Run the program in interactive mode. The shorthand \texttt{-i} can be used instead of \texttt{-{}-interactive}.
\end{itemize}
Additional options:
\begin{itemize}
\item \texttt{-{}-default}: Generates the file, entirely using default values, such as generic error messages, instead of user input. Can, e.g., be used to quickly check the generated context rule(s).
\item \texttt{-o <filename>}: Saves the generated file to \texttt{<filename>} instead of printing it to standard output.
\item \texttt{-{}-no-output}: Generates the file, but doesn't output the result.
\item \texttt{-help, -{}-help}: Displays the list of options.
\end{itemize}
\bigskip

\subsection{User input}
When generating a script, \texttt{sgen} might need some extra information from the user. There are two kinds of information:
\begin{itemize}
\item \textbf{Preface}: information to go at the beginning of the script. Contains metainformation about the script such as description, author, etc. See the full list in Section \ref{config}.
\item \textbf{Rule information}: rule-specific info, such as error messages that are output in \texttt{org} and \texttt{report} mode.
\end{itemize}
There are two ways of passing this information, interactive mode and configuration mode.
\bigskip

\subsubsection{Interactive mode\label{interact}}
In interactive mode, the program prompts the user for the information through the commandline. The user can then choose to save the information into an \texttt{sgen} config file, which can be further modified and reused in configuration mode.
\bigskip

\subsubsection{Configuration mode\label{config}}
In configuration mode, the program looks for an \texttt{sgen} config file to provide the needed information to generate the file. For \texttt{<file>}.cocci, the config file should be called \texttt{<file>}.config.\\\\
The user-specified attributes in the \textbf{preface} are, in no specific order ((r) means required):
\begin{center}
\renewcommand{\arraystretch}{1.2}
\begin{tabular}{p{3cm}p{2.1cm}p{3.6cm}p{5.5cm}}
\textbf{Attribute name} & \textbf{Shorthand} & \textbf{Value} & \textbf{Description}\\
\texttt{description} (r) & \texttt{d} & Single-value text & Describes what the Coccinelle script does.\\
\texttt{confidence} (r) & \texttt{c} & Low, Moderate, High & Confidence level for the script.\\
\texttt{authors} & \texttt{a} & Multi-value text & Authors of the script, including affiliation and license.\\
\texttt{url} & \texttt{u} & Single-value text & URL for the script.\\
\texttt{limitations} & \texttt{l} & Multi-value text & Limitations for the script.\\
\texttt{keywords} & \texttt{k} & Single-value text & Keywords for the script.\\
\texttt{options} & \texttt{o} & Single-value text &\texttt{spatch} options with which to run the script.\\
\texttt{comments} & \texttt{m} & Single-value text & Additional comments.\\
\end{tabular}
\end{center}\vspace{0.5cm}
The \textbf{rule information} contains error messages for \texttt{org} and \texttt{report}, and possibly rule names for rules that are unnamed in the original \texttt{Coccinelle script}.\\\\
\clearpage
\noindent The syntax for the \texttt{sgen} config files is rather simple, and the easiest way to learn it is to run \texttt{sgen} in interactive mode and study the resulting config. But for completeness ...:
\begin{itemize}
\item The syntax for attributes is
\begin{verbatim}
 <attribute_name> = <value>
\end{verbatim}
where \texttt{<attribute\_name>} can be either the attribute name or its shorthand. The end of \texttt{<value>} is marked by a newline. It is therefore not possible to insert newlines in any of the values.
\item For multi-valued attributes, values are delimited by pipes, \texttt{|}, ie.
\begin{verbatim}
<attribute_name> = <value_1>|<value_2>|...|<value_n>
\end{verbatim}
\item Error messages for rules follow the syntax
\begin{verbatim}
<rule_name> =
  org:<message>
  report:<message>
\end{verbatim}
for \texttt{org} and \texttt{report} error messages, respectively. Here, \texttt{<rule\_name>} is either the actual rule name, or, if it is a nameless rule, \texttt{<line\_that\_rule\_starts\_on>:<new\_name>}.\newline
Meanwhile, \texttt{<message>} follow the syntax of \texttt{python} format strings, e.g.
\begin{verbatim}
"This is a message that references two metavariables %s and %s." % (x,y)
\end{verbatim}
where \texttt{x} and \texttt{y} are metavariables in the rule. If using metavariables from another rule, write \texttt{<other\_rule\_name>.<metavar\_name>}. If using no metavariables, just write the error message surrounded by quotes.
\item Comments can be written in \texttt{C}-style, ie. \texttt{//} and \texttt{/**/}.
\end{itemize}
\bigskip

\subsection{Examples}
Example files can be found in the \texttt{examples} directory. For each example, there should be four files. The file extensions denote the following:
\begin{itemize}
\item \texttt{<name>.c}: \texttt{C} source file that returns matches/patches for the corresponding cocci file. Can be tested with \texttt{spatch --sp-file <name>.cocci <name>.c}.
\item \texttt{<name>.cocci}: simple, unhardened Coccinelle script.
\item \texttt{<name>.config}: \texttt{sgen} configuration file for specifying preface and rule information.
\item \texttt{<name>\_.cocci}: expected output when running \texttt{sgen} on the unhardened Coccinelle script with the config file. Should be a valid, hardened Coccinelle script. Can be tested with e.g. \texttt{spatch --sp-file <name>\_.cocci <name>.c -D report --no-show-diff}.
\end{itemize}

%PSEUDOGRAMMAR:
%\begin{center}
%\renewcommand{\arraystretch}{1.2}
%\begin{tabular}{p{3cm}p{11cm}}
%config : & (declaration) NL (config) \newline EOF \\
%declaration: & (attribute) = (value) \newline
%               (attribute) = (multivalue) \newline
%               (rulename) = (messages) \\
%attribute : & description \newline
%              comments ... \\
%value : & string (no newline) \\
%multivalue : & value | multivalue \newline
%               value \\
%rulename : & string (rulename) \newline
%             int : string (line, new rule) \\
%messages : & org : (message) NL report : (message) \newline
%             org : (message) \newline
%             report : (message) \\
%message : & "string" \newline "string" \% metavariables
%\end{tabular}
%\end{center}

\clearpage

\section{Implementation}
This section contains information about the implementation details of the tool.
\\TODO ... 
%Some notes for later:
%\begin{itemize}
%\item Implementation overview (showing module dependencies and the program flow).
%\item Extensive use of the AST0 visitor. Reason: abstracts away a lot of boilerplate code for accessing the components of the abstract syntax tree.
%\item Easy way to debug in \texttt{rule\_body.ml}: add \texttt{>> GT.add "debug msg" >>} in some function sequence.
%\item Generator types, in particular snapshot and the combiner type in rule body (which is probably the most complex part of the implementation).
%\item The generating mode flag: cannot be substituted for ignore patch or match since that option means that the plus and minus tree are not processed in the same way.
%\end{itemize}





\clearpage

\section{Known issues}
This section lists the known issues that might either cause the tool to fail or to generate an erroneous script.
\begin{itemize}
\item \textbf{Missing information in rule headers}: Rule headers will not be generated correctly if the original rules contain \texttt{extends} or \texttt{expression}. Those qualifiers will be missing in the generated rule since they are not included in the output of the parser.\newline
Fix: Change the parser to include this information.
\item \textbf{Typedefs in rule headers}: If there are meta typedefs in the original rule headers, they will be included in every generated rule that uses the type. This causes an error when using the generated script since meta typedefs can only be declared once.\newline
Fix: Remove the error check in the parser since this should not cause an error.
\item Disjunction generation has a number of issues:
\begin{itemize}
  \item \textbf{Selecting wrong position in statement dots cases}: There must be the same number of positions in each disjunction case, otherwise \texttt{org} and \texttt{report} will only match when all positions can be found. In statement dots disjunctions, this is currently solved by putting the position at the first possible statement. The issue here is that, if the case contains many statements, only the first surrounding statement will be highlighted instead of the important part.\newline
Fix: Implement support for finding a single best statement in a statement dots (list of statements).
  \item \textbf{Nested disjunctions}: The position counter is frozen within a disjunction. But if there is a nested disjunction inside it, the same position will be used in both disjunction levels, causing a nonsensical script.\newline
Fix: Keep track of the current nest and name the position accordingly.
\end{itemize}
\item \textbf{No format string metavariable check}: The user can specify metavariables to be used in the messages for \texttt{org} and \texttt{report} mode. The program currently does not check if the declared metavariables actually exist in the original rule.\newline
Fix: Implement check of metavariables.
\item \textbf{No special-character rulename check}: In \texttt{Coccinelle}, allowed rulenames are limited to allowed identifiers in the \texttt{C} language, ie. certain characters are not allowed. In \texttt{sgen} when naming nameless rules, any rulename that was not already used/does not contain whitespaces will be accepted.\newline
Fix: Limit the allowed user-specified rulenames to \texttt{C} identifiers.
\item \textbf{Dependencies between patch rules}: It is possible to make patch rules that depend on other patch rules modifying the code. E.g. if one patch rule transforms f(0) and one transforms f(e), then f(0) will only match the first, since it is transformed to something else when it reaches the f(e) rule. But in \texttt{context} mode, both rules will print the f(0) occurrence.\newline
Fix: ??? Somehow detect that two rules will match the same case and insert constraints such that any match in subsequent rules do not match the first one.
\item \textbf{Dependencies in context rules}: In a \texttt{context} script, if there are dependencies between rules, they might be mixed up. This happens if there are rules dependent on the generated rules since they will then be printed before the rules on which they are dependent!\newline
Fix: ???
\item \textbf{Type and switch case disjunctions}: Currently, the program fails if attempting to generate a \texttt{Coccinelle} script with type or switch case disjunctions. The failure happens in the position generator. The reason it is not implemented is that it requires quite a lot of code for a case that rarely appears.\newline
Fix: Implement full position generation for types and switch cases.
\end{itemize}

\clearpage

\section{Future work}
This section lists the work to be done on the tool aside from fixing the known issues.
\begin{itemize}
\item \textbf{Global configuration file}: Some of the preface attributes remain more or less constant across generated scripts, such as author and url. Furthermore, there are some hardcoded values in the program that should be configurable, such as the default values for rulenames, position names, and error messages.\newline
It could be useful to have some kind of configuration file that collects these values and reuses them each time a script is generated.
\item \textbf{Test framework}: There are currently no tests for the tool. It should be possible to use the \texttt{Coccinelle} test framework. Tests could be something like checking if the generated scripts are syntactically correct, ie. parsable. It is however not easy to test correctness of e.g. the positions and stars since there is more than one way to do that correctly.
\item \textbf{Generic rule splitting}: The disjunction rule generation implementation currently doesn't lend itself very well to possible expansion. For instance, if there was another case where we would like to expand a rule into several rules, how could we do this in a generalised way? How should this work with several split rules?\newline
Example: any construct where the patch rule changes something other than itself, e.g. function declarations; if the declared function has a prototype, the prototype is changed as well in \texttt{patch} mode, but this has to be done explicitly with two rules in \texttt{context} mode.\newline
Discussion: consider splitting generation of extra rules out into its own module, one for each type of split rule. Essentially using multiple passes over the AST0 to get the rules, one pass per rule type. This is however slightly complicated by the fact that context rule generation should be modified if there is a disjunction.
\item \textbf{Stars and braces}: If a braced statement is starred, it would be nicer if the braces were not on the same lines as the stars.
\item \textbf{Character limit}: Ensuring character limit in the generated rule. This is currently implemented for the preface, but not for rule headers and script rules.
\item Perhaps rethink position generation at some point. If the script already contains minuses, we would rather put the positions there compared to the heuristic version.
\end{itemize}


\end{document}
