\documentclass{article}
\usepackage{hevea}
\usepackage{fullpage}
\usepackage{alltt}
\usepackage{xspace}
\usepackage[pdfborder={0 0 0}]{hyperref}
\usepackage{listings}
\usepackage[usenames,dvipsnames]{color}
\usepackage[T1]{fontenc}

\lstset{basicstyle=\ttfamily,numbers=left, numberstyle=\tiny, stepnumber=1, numbersep=5pt,language=C,commentstyle=\color{OliveGreen},keywordstyle=\color{blue},stringstyle=\color{BrickRed}}

% You must prefix the +/- lines of
% cocci files with @+/@- respectively.
% This will enable the automatic coloration.
\ifhevea
\lstdefinelanguage{Cocci}{
morekeywords={when,strict,any,forall,exists},
keywordstyle=\color{Bittersweet}\bfseries,
sensitive=false
}
\else
\lstdefinelanguage{Cocci}{
morekeywords={idexpression,expression,statement,identifier,
parameter,list,when,strict,any,forall,exists},
keywordstyle=\color{Bittersweet}\bfseries,
sensitive=false,
morecomment=*[s][\color{OliveGreen}]{@}{@@},
morecomment=*[s][\color{OliveGreen}]{@@}{@@},
moredelim=[il][\color{blue}]{@+},
moredelim=[il][\color{BrickRed}]{@-}
}
\fi

\newif\iflanguagestyle
\languagestylefalse
% Definition of a grammar (BNF style) package for Latex and Hevea


\ifhevea
% Definition for Hevea (HTML generation)
\def\T#1{{\sf{#1}}}
\def\NTS#1{{\maroon #1\/}}
\def\KW#1{{\blue #1}}
\def\gramor{{\black $|$}}
\def\grameq{{\black \quad::=\quad}}
\def\lparen{{\black (}}
\def\rparen{{\black )}}
\def\lbracket{{\black [}}
\def\rbracket{{\black ]}}
\def\plus{{\black +}}
\def\questionmark{{\black ?}}
\def\etoile{{\black *}}
\else
% Definition for LaTeX
\def\T#1{{\textsf{\small{#1}}}}
\def\NTS#1{{\it #1\/}}
\def\KW#1{{\mtt{#1}}}
%\def\gramor{$\vert$}
\def\gramor{$\mid$}
\def\grameq{\,\,\,::=\,\,\,\,\,}
\def\lparen{(}
\def\rparen{)}
\def\lbracket{$[$}
\def\rbracket{$]$}
\def\plus{+}
\def\questionmark{?}
\def\etoile{*}
\fi

\def\NT#1{\hyperlink{#1}{\NTS{#1}}}
\def\group#1{{\rm\lparen}#1{\rm\rparen}}
\def\range#1#2{#1{..}#2}
\def\any#1{#1$^{\etoile}$}
\def\some#1{#1$^{\plus}$}
\def\ANY#1{\any{{\rm\lparen}#1{\rm\rparen}}}
\def\SOME#1{\some{{\rm\lparen}#1{\rm\rparen}}}
\def\OR{\gramor\ }

\iflanguagestyle
% Option notation : [ xxx ] versus (xxx)^?
\def\opt#1{#1$^{\questionmark}$}
\def\OPT#1{\opt{{\rm\lparen}#1{\rm\rparen}}}
\else
\def\opt#1{{\lbracket}#1{\rbracket}}
\def\OPT#1{\opt{#1}}
\fi

\newenvironment{grammar}{\begin{center}\begin{tabular}{l@{}c@{}l}}{\end{tabular}\end{center}}
\def\RULE#1\CASE#2{\NTS{#1} & \grameq & \KW{#2} \\}
\def\CASE#1{& \gramor & \KW{#1} \\}

\newcommand{\rt}[1]{\hypertarget{#1}{#1}}
\newcommand{\bs}{\textbackslash}

\def\lb{\char123}
\def\rb{\char125}
\def\lt{\tt\char60}
\def\gt{\tt\char62}
\def\caret{\tt\^{}}


\newcommand{\sizecodebis}[0]{\scriptsize}

\newcommand{\mita}[1]{\mbox{\it{{#1}}}}
\newcommand{\mtt}[1]{\mbox{\tt{{#1}}}}
\newcommand{\msf}[1]{\mbox{\sf{{#1}}}}
\newcommand{\stt}[1]{\mbox{\scriptsize\tt{{#1}}}}
\newcommand{\ssf}[1]{\mbox{\scriptsize\sf{{#1}}}}
\newcommand{\sita}[1]{\mbox{\scriptsize\it{{#1}}}}
\newcommand{\mrm}[1]{\mbox{\rm{{#1}}}}
\newcommand{\mth}[1]{\({#1}\)}
\newcommand{\entails}[2]{\begin{array}{@{}c@{}}{#1}\\\hline{#2}\end{array}}
\newcommand{\ttlb}{\mbox{\tt \char'173}}
\newcommand{\ttrb}{\mbox{\tt \char'175}}
\newcommand{\ttmid}{\mbox{\tt \char'174}}
\newcommand{\tttld}{\mbox{\tt \char'176}}

\newcommand{\fixme}[1]{{\color{red} #1}}

\ifhevea
\newcommand{\phantom}{}
\newcommand{\air}{   }
\else
\newcommand{\air}{\phantom{xxx}}
\fi

\title{The SmPL Grammar}
\author{Research group on Coccinelle}
\date{\today}

\begin{document}
\maketitle

%\section{The SmPL Grammar}

% This section presents the SmPL grammar.  This definition follows closely
% our implementation using the Menhir parser generator \cite{menhir}.

The grammar uses some rules where the left-hand side is in all capital
letters.  These are macros, which take one or more grammar rule
right-hand-sides as arguments.  The grammar also uses some unspecified
nonterminals, such as {\sf id}, {\sf const}, etc.  These refer to the
sets suggested by the name, {\em i.e.}, {\sf id} refers to the set of
possible C-language identifiers, while {\sf const} refers to the set
of possible C-language constants. \ifhevea A PDF version of this
documention is available at
\url{http://localhost:8080/coccinelle/cocci_syntax.pdf}.  \fi

\section{Program}

\begin{grammar}
  \RULE{\rt{program}}
  \CASE{\any{\NT{include}} \some{\NT{changeset}}}

  \RULE{\rt{include}}
  \CASE{using "\NT{string}"}
  \CASE{using <\NT{pathToIsoFile}>}

  \RULE{\rt{changeset}}
  \CASE{\NT{metavariables} \ANY{--- filename +++ filename} \NT{transformation}}

\end{grammar}

Between the metavariables and the transformation rule, there can be a
specification of constraints on the names of the old and new files,
analogous to the filename specifications in the standard patch syntax.
%(see Figure \ref{scsiglue_patch}).

\section{Metavariables}

Fresh metavariables must only be used in {\tt +} code.  Metavariables must
occur at least once in the transformation immediately following their
declaration.  These properties are not expressed in the grammar, but are
checked by a subsequent analysis.

\begin{grammar}
  \RULE{\rt{metavariables}}
  \CASE{@@ \any{\NT{metadecl}} @@}
  \CASE{@ \NT{rulename} @ \any{\NT{metadecl}} @@}

  \RULE{\rt{rulename}}
  \CASE{\T{id} \OPT{extends \T{id}} \OPT{depends on \NT{dep}} \opt{\NT{iso}} \opt{\NT{disable}} \opt{\NT{exists}} \opt{expression}}
  \CASE{script:\T{language} \OPT{depends on \NT{dep}}}

  \RULE{\rt{dep}}
  \CASE{\NT{pnrule}}
  \CASE{\NT{dep} \&\& \NT{dep}}
  \CASE{\NT{dep} || \NT{dep}}

  \RULE{\rt{pnrule}}
  \CASE{\T{id}}
  \CASE{!\T{id}}
  \CASE{ever \T{id}}
  \CASE{never \T{id}}
  \CASE{(\NT{dep})}

  \RULE{\rt{iso}}
  \CASE{using "\T{str}" \ANY{, "\T{str}"}}

  \RULE{\rt{disable}}
  \CASE{disable \T{id} \ANY{, \T{id}}}

  \RULE{\rt{exists}}
  \CASE{exists}
  \CASE{\opt{reverse} forall}
\end{grammar}

\begin{grammar}
  \RULE{\rt{metadecl}}
  \CASE{fresh identifier \NT{ids} ;}
  \CASE{parameter \opt{list} \NT{ids} ;}
  \CASE{expression list \NT{ids} ;}
  \CASE{type \NT{ids} ;}
  \CASE{statement \opt{list} \NT{ids} ;}
  \CASE{typedef \NT{ids} ;}
  \CASE{declarer name \NT{ids} ;}
  \CASE{iterator name \NT{ids} ;}
  \CASE{identifier \NT{pmid\_with\_not\_eq\_list} ;}
  \CASE{\opt{local} function \NT{pmid\_with\_not\_eq\_list} ;}
  \CASE{declarer \NT{pmid\_with\_not\_eq\_list} ;}
  \CASE{iterator \NT{pmid\_with\_not\_eq\_list} ;}
  \CASE{error \NT{pmid\_with\_not\_eq\_list} ; }
  \CASE{\opt{local} idexpression \opt{\NT{ctype}} \NT{pmid\_with\_not\_eq\_list} ;}
  \CASE{\opt{local} idexpression \OPT{\ttlb \NT{ctypes} \ttrb \any{*}} \NT{pmid\_with\_not\_eq\_list} ;}
  \CASE{\opt{local} idexpression \some{*} \NT{pmid\_with\_not\_eq\_list} ;}
  \CASE{expression \some{*} \NT{pmid\_with\_not\_eq\_list} ;}
  \CASE{\NT{ctype} [ ] \NT{pmid\_with\_not\_eq\_list} ;}
  \CASE{\ttlb \NT{ctypes} \ttrb \any{*} [ ] \NT{pmid\_with\_not\_eq\_list} ;}
  \CASE{constant \opt{\NT{ctype}} \NT{pmid\_with\_not\_eq\_list} ;}
  \CASE{constant \OPT{\ttlb \NT{ctypes} \ttrb \any{*}} \NT{pmid\_with\_not\_eq\_list} ;}
  \CASE{expression \NT{pmid\_with\_not\_ceq\_list} ;}
  \CASE{\NT{ctype} \NT{pmid\_with\_not\_ceq\_list} ;}
  \CASE{\ttlb \NT{ctypes} \ttrb \any{*} \NT{pmid\_with\_not\_ceq\_list} ;}
  \CASE{position \opt{any} \NT{pmid\_with\_not\_eq\_mid\_list} ;}
  \CASE{parameter list [ ident ] \NT{ids} ;}
  \CASE{expression list [ ident ] \NT{ids} ;}
\end{grammar}

\begin{grammar}
  \RULE{\rt{ids}}
  \CASE{\NT{pmid} \ANY{, \NT{pmid}}}

  \RULE{\rt{pmid}}
  \CASE{\T{id}}
  \CASE{\NT{mid}}
  \CASE{list}
  \CASE{error}
  \CASE{type}

  \RULE{\rt{mid}}  \CASE{\T{rulename\_id}.\T{id}}

  \RULE{\rt{pmid\_with\_not\_eq\_list}}
  \CASE{\NT{pmid\_with\_not\_eq} \ANY{, \NT{pmid\_with\_not\_eq}}}

  \RULE{\rt{pmid\_with\_not\_eq}}
  \CASE{\NT{pmid} \OPT{!= \T{id}}}
  \CASE{\NT{pmid} \OPT{!= \ttlb \T{id} \ANY{, \T{id}} \ttrb}}

  \RULE{\rt{pmid\_with\_not\_ceq\_list}}
  \CASE{\NT{pmid\_with\_not\_ceq} \ANY{, \NT{pmid\_with\_not\_ceq}}}

  \RULE{\rt{pmid\_with\_not\_ceq}}
  \CASE{\NT{pmid} \OPT{!= \NT{id\_or\_cst}}}
  \CASE{\NT{pmid} \OPT{!= \ttlb \NT{id\_or\_cst} \ANY{, \NT{id\_or\_cst}} \ttrb}}

  \RULE{\rt{id\_or\_cst}}
  \CASE{\T{id}}
  \CASE{\T{integer}}

  \RULE{\rt{pmid\_with\_not\_eq\_mid\_list}}
  \CASE{\NT{pmid\_with\_not\_eq\_mid} \ANY{, \NT{pmid\_with\_not\_eq\_mid}}}

  \RULE{\rt{pmid\_with\_not\_eq\_mid}}
  \CASE{\NT{pmid} \OPT{!= \NT{mid}}}
  \CASE{\NT{pmid} \OPT{!= \ttlb \NT{mid} \ANY{, \NT{mid}} \ttrb}}
\end{grammar}

Subsequently, we refer to arbitrary metavariables as
\mth{\msf{metaid}^{\mbox{\scriptsize{\it{ty}}}}}, where {\it{ty}} indicates
the {\it metakind} used in the declaration of the variable.  For example,
\mth{\msf{metaid}^{\ssf{Type}}} refers to a metavariable that stands for
any type.

The {\it type} nonterminal is used by both the grammar of metavariable
declarations and the grammar of transformations, and is defined on
page~\pageref{types}.

\section{Transformation}

The grammar of the transformation is not actually the grammar of the SmPL
code that can be written by the programmer, but the grammar of the slice of
this consisting of the {\tt -} annotated and the unannotated code (the
context of the transformed lines), or the {\tt +} annotated code and the
unannotated code.  For example, for parsing purposes, the transformation
%presented in Section \ref{sec:seq2}
is split into the two variants shown below and each is parsed
separately.

\begin{center}
\begin{tabular}{c}
\begin{lstlisting}[language=Cocci]
  proc_info_func(...) {
    <...
@--    hostno
@++    hostptr->host_no
    ...>
 }
\end{lstlisting}\\
\end{tabular}
\end{center}

{%\sizecodebis
\begin{center}
\begin{tabular}{p{5cm}p{3cm}p{5cm}}
\begin{lstlisting}[language=Cocci]
  proc_info_func(...) {
    <...
@--    hostno
    ...>
 }
\end{lstlisting}
&&
\begin{lstlisting}[language=Cocci]
  proc_info_func(...) {
    <...
@++    hostptr->host_no
    ...>
 }
\end{lstlisting}
\end{tabular}
\end{center}
}

\noindent
Requiring that both slices parse correctly ensures that the rule matches
syntactically valid C code and that it produces syntactically valid C code.
The generated parse trees are then merged for use in the subsequent
matching and transformation process.

The grammar rule for the minus or plus slice of a transformation is as follows:

\begin{grammar}

  \RULE{\rt{transformation}}
  \CASE{\NT{fundecl}}
  \CASE{\NT{ctype}}
  \CASE{\ttlb \NT{initialize\_list} \ttrb}
  \CASE{\NT{toplevel\_seq\_start\_after\_dots\_init}}

  \RULE{\rt{toplevel\_seq\_start\_after\_dots\_init}}
  \CASE{\NT{stmt\_dots} \NT{toplevel\_after\_dots}}
  \CASE{\NT{expr} \opt{\NT{toplevel\_after\_exp}}}
  \CASE{\NT{decl\_stmt\_expr} \opt{\NT{toplevel\_after\_stmt}}}

  \RULE{\rt{stmt\_dots}}
  \CASE{... \any{\NT{whenppdecls}}}
  \CASE{<... \any{\NT{whenppdecls}} \NT{nest\_after\_dots} ...>}
  \CASE{<+... \any{\NT{whenppdecls}} \NT{nest\_after\_dots} ...+>}

  \RULE{\rt{whenppdecls}}
  \CASE{when != \NT{when\_start}$^\dag$}
  \CASE{when = \NT{rule\_elem\_stmt}$^\dag$}
  \CASE{when \NT{any\_strict} \ANY{, \NT{any\_strict}} $^\dag$}
  \CASE{when true != \NT{exp} $^\ddag$}
  \CASE{when false != \NT{exp} $^\ddag$}

  \RULE{\rt{any\_strict}}
  \CASE{any}
  \CASE{strict}
  \CASE{forall}
  \CASE{exists}

  \RULE{\rt{nest\_after\_dots}}
  \CASE{\NT{decl\_stmt\_exp} \opt{\NT{nest\_after\_stmt}}}
  \CASE{\opt{\NT{exp}} \opt{\NT{nest\_after\_exp}}}

  \RULE{\rt{nest\_after\_stmt}}
  \CASE{\NT{stmt\_dots} \NT{nest\_after\_dots}}
  \CASE{\NT{decl\_stmt} \opt{\NT{nest\_after\_stmt}}}

  \RULE{\rt{nest\_after\_exp}}
  \CASE{\NT{stmt\_dots} \NT{nest\_after\_dots}}

  \RULE{\rt{toplevel\_after\_dots}}
  \CASE{\opt{\NT{toplevel\_after\_exp}}}
  \CASE{\NT{exp} \opt{\NT{toplevel\_after\_exp}}}
  \CASE{\NT{decl\_stmt\_expr} \NT{toplevel\_after\_stmt}}

  \RULE{\rt{toplevel\_after\_exp}}
  \CASE{\NT{stmt\_dots} \opt{\NT{toplevel\_after\_dots}}}

  \RULE{\rt{decl\_stmt\_expr}}
  \CASE{TMetaStmList$^\ddag$}
  \CASE{\NT{decl\_var}}
  \CASE{\NT{stmt}}
  \CASE{(\NT{fun\_start} \ANY{| \NT{fun\_start}})}

  \RULE{\rt{toplevel\_after\_stmt}}
  \CASE{\NT{stmt\_dots} \opt{\NT{toplevel\_after\_dots}}}
  \CASE{\NT{decl\_stmt} \NT{toplevel\_after\_stmt}}

\end{grammar}

$^\dag$ Note
$^\ddag$ Check and fix me

% \noindent{\footnotesize\begin{tabular}{r@{\,\,\,}c@{\,\,\,}l}
% \mita{transformation} & ::= &
%    \begin{tabular}[t]{@{}l}
%    \ANY{\mtt{\#include} \msf{include\_string}} \\
%    \opt{OPTDOTSEQ(\some{\mita{fun\_decl\_statement}} \(\mid\) \mita{expr},
%    \mita{stmt\_whencode})}\end{tabular} \\
% \mita{fun\_decl\_statement} & ::= & \mita{decl\_statement} \(\mid\)
% \mita{fun\_decl}
% \end{tabular}

% \begin{grammar}
%   \PRULE{OPTDOTSEQ(\NT{grammar},\NT{whencode})}
%   \CASE{\opt{... \opt{\NT{whencode}}} \NT{grammar}
%  \ANY{... \opt{\NT{whencode}} \NT{grammar}}
%  \opt{... \opt{\NT{whencode}}}}
% \end{grammar}

\noindent
Lines may be annotated with an element of the set $\{\mtt{-},
\mtt{+}\}$ or an element of the set $\{\mtt{*}, \mtt{?}\}$, or one of
each. \mtt{?} and \mtt{*} represent respectively at most one, and at
least one match of the given pattern.  There are some constraints on
the use of these annotations:
\begin{itemize}
\item Dots, {\em i.e.} \texttt{...}, cannot occur on a line marked
  \texttt{+}.
\item Nested dots, {\em i.e.} dots enclosed in {\tt <} and {\tt >}, cannot
  occur on a line with any marking.
\end{itemize}


\section{Types}
\label{types}

\begin{grammar}

  \RULE{\rt{ctypes}}
  \CASE{\NT{ctype} \ANY{, \NT{ctype}}}

  \RULE{\rt{ctype}}
  \CASE{\opt{\NT{const\_vol}} \NT{generic\_ctype} \any{*}}
  \CASE{\opt{\NT{const\_vol}} void \some{*}}
  \CASE{(\NT{ctype} \ANY{| \NT{ctype}})}

  \RULE{\rt{const\_vol}}
  \CASE{const}
  \CASE{volatile}

  \RULE{\rt{generic\_ctype}}
  \CASE{\NT{ctype\_qualif}}
  \CASE{\opt{\NT{ctype\_qualif}} char}
  \CASE{\opt{\NT{ctype\_qualif}} short}
  \CASE{\opt{\NT{ctype\_qualif}} int}
  \CASE{\opt{\NT{ctype\_qualif}} long}
  \CASE{double}
  \CASE{float}
  \CASE{\OPT{struct\OR union} \T{id} \OPT{\{ \any{\NT{struct\_decl\_list}} \}}}

  \RULE{\rt{ctype\_qualif}}
  \CASE{unsigned}
  \CASE{signed}

  \RULE{\rt{struct\_decl\_list}}
  \CASE{\NT{struct\_decl\_list\_start}}

  \RULE{\rt{struct\_decl\_list\_start}}
  \CASE{\NT{struct\_decl}}
  \CASE{\NT{struct\_decl} \NT{struct\_decl\_list\_start}}
  \CASE{... \opt{when != \NT{struct\_decl}}$^\dag$ \opt{\NT{continue\_struct\_decl\_list}}}

  \RULE{\rt{continue\_struct\_decl\_list}}
  \CASE{\NT{struct\_decl} \NT{struct\_decl\_list\_start}}
  \CASE{\NT{struct\_decl}}

  \RULE{\rt{struct\_decl}}
  \CASE{\NT{ctype} \NT{d\_ident};}
  \CASE{\NT{fn\_ctype} (* \NT{d\_ident}) (\NT{});)}
  \CASE{\opt{\NT{const\_vol}} \NT{pure\_ident} \NT{d\_ident};}

  \RULE{\rt{d\_ident}}
  \CASE{\NT{ident} \any{[\opt{\NT{eexpr}}]}}

  \RULE{\rt{fn\_ctype}}
  \CASE{\NT{generic\_ctype} \any{*}}
  \CASE{void \any{*}}
\end{grammar}

$^\dag$ The optional \texttt{when} construct ends at the end of the line.

% \noindent{\footnotesize\begin{tabular}{r@{\,\,\,}c@{\,\,\,}l}
% \mita{type} & ::= & \opt{\mtt{const} \(\mid\) \mtt{volatile}}
%                     \mita{type\_desc} \ANY{\mtt{*}}\\
% \mita{type\_desc} & ::= & \mita{simple\_type} \(\mid\) \opt{\mtt{signed} \(\mid\)
%        \mtt{unsigned}} \mita{signable\_type} \(\mid\) \opt{\mtt{struct} \(\mid\)
%        \mtt{union}} \msf{id}
% \\&\multicolumn{1}{r}{\(\mid\)}&
%        \mth{\msf{metaid}^{\ssf{Type}}}
% \end{tabular}}

% \noindent{\footnotesize\begin{tabular}{r@{\,\,\,}c@{\,\,\,}l}
% \mita{simple\_type} & ::= & \mtt{void} \(\mid\) \mtt{double} \(\mid\)
%  \mtt{float} \\
% \mita{signable\_type} & ::= & \mtt{char} \(\mid\) \mtt{short} \(\mid\)
%  \mtt{int} \(\mid\) \mtt{long}
% \end{tabular}}

\section{Function declarations}

\begin{grammar}

  \RULE{\rt{fundecl}}
  \CASE{\opt{static} \NT{funid}
    (\opt{\NT{PARAMSEQ}(\NT{param},\mth{\varepsilon})})
    \ttlb~\opt{\NT{stmt\_seq}} \ttrb}

  \RULE{\rt{funid}}
  \CASE{\T{id}}
  \CASE{\mth{\T{metaid}^{\ssf{Func}}}}
  \CASE{\mth{\T{metaid}^{\ssf{LocalFunc}}}}

  \RULE{\rt{param}}
  \CASE{\NT{type} \T{id}}
  \CASE{\mth{\T{metaid}^{\ssf{Param}}}}
  \CASE{\mth{\T{metaid}^{\ssf{ParamList}}}}
\end{grammar}

\begin{grammar}
  \RULE{\rt{PARAMSEQ}(\NT{grammar},\NT{whencode})}
  \CASE{\mth{(}\NT{grammar}\OR \ldots \opt{\NT{whencode}}\mth{)} \ANY{, \NT{grammar}\OR , \ldots \opt{\NT{whencode}}}}
\end{grammar}

%\newpage

\section{Declarations}

\begin{grammar}
  \RULE{\rt{decl}}
  \CASE{\NT{type} \opt{\NT{id} \opt{[\opt{\NT{dot\_expr}}]}
      \ANY{, \NT{id} \opt{[ \opt{\NT{dot\_expr}}]}}};}
  \CASE{\NT{type} \NT{id} \opt{[\opt{\NT{dot\_expr}}]}= \NT{nest\_expr};}
\end{grammar}

% \noindent{\footnotesize\begin{tabular}{r@{\,\,\,}c@{\,\,\,}l}
% \mita{decl\_var} & ::= &
%    \mita{type} \opt{\ANY{\mita{id} \opt{\mtt{[} \opt{\mita{dot\_expr}}
%    \mtt{]}} \mtt{,}} \mita{id} \opt{\mtt{[} \opt{\mita{dot\_expr}} \mtt{]}}}
%    \mtt{;}
% \\&\multicolumn{1}{r}{\(\mid\)}&
%  \mita{type} \mita{id} \opt{\mtt{[} \opt{\mita{dot\_expr}} \mtt{]}}
%    \mtt{=} \mita{nest\_expr} \mtt{;}
% \end{tabular}}

\section{Statements}

The first rule {\em statement} describes the various forms of a statement.
The remaining rules implement the constraints that are sensitive to the
context in which the statement occurs: {\em single\_statement} for a
context in which only one statement is allowed, and {\em decl\_statement}
for a context in which a declaration, statement, or sequence thereof is
allowed.

%\vspace{\baselineskip}
\begin{grammar}
  \RULE{\rt{stmt}}
  \CASE{\NT{includes}}
  \CASE{\mth{\T{metaid}^{\ssf{Stmt}}}}
  \CASE{\NT{expr};}
  \CASE{if (\NT{dot\_expr}) \NT{single\_stmt} \opt{else \NT{single\_stmt}}}
  \CASE{for (\opt{\NT{dot\_expr}}; \opt{\NT{dot\_expr}}; \opt{\NT{dot\_expr}})
    \NT{single\_stmt}}
  \CASE{while (\NT{dot\_expr}) \NT{single\_stmt}}
  \CASE{do \NT{single\_stmt} while (\NT{dot\_expr});}
  \CASE{\NT{iter\_ident} (\any{\NT{dot\_expr}}) \NT{single\_stmt}}
  \CASE{switch (\opt{\NT{dot\_expr}}) \ttlb \any{\NT{case\_line}} \ttrb}
  \CASE{return \opt{\NT{dot\_expr}};}
  \CASE{\ttlb~\opt{\NT{stmt\_seq}} \ttrb}
  \CASE{\NT{NEST}(\some{\NT{decl\_stmt}}, \NT{stmt\_whencode})}
  \CASE{\NT{NEST}(\NT{expr}, \NT{stmt\_whencode})}
  \CASE{break;}
  \CASE{continue;}
  \CASE{\NT{ident}:}
  \CASE{goto \NT{ident};}
  \CASE{\ttlb \NT{fun\_start} \ttrb}

% \noindent{\footnotesize\begin{tabular}{r@{\,\,\,}c@{\,\,\,}l}
% \mita{statement} & ::= &
%   \mth{\msf{metaid}^{\ssf{Stmt}}}
% \\&\multicolumn{1}{r}{\(\mid\)}&
%   \mita{expr} \mtt{;}
% \\&\multicolumn{1}{r}{\(\mid\)}&
%   \mtt{if} \mtt{(} \mita{dot\_expr} \mtt{)} \mita{single\_statement}
%   \opt{\mtt{else} \mita{single\_statement}}
% \\&\multicolumn{1}{r}{\(\mid\)}&
%   \mtt{for} \mtt{(} \opt{\mita{dot\_expr}} \mtt{;} \opt{\mita{dot\_expr}} \mtt{;}
%   \opt{\mita{dot\_expr}} \mtt{)} \mita{single\_statement}
% \\&\multicolumn{1}{r}{\(\mid\)}&
%   \mtt{while} \mtt{(} \mita{dot\_expr} \mtt{)} \mita{single\_statement}
% \\&\multicolumn{1}{r}{\(\mid\)}&
%   \mtt{do} \mita{single\_statement} \mtt{while} \mtt{(} \mita{dot\_expr} \mtt{)}
%   \mtt{;}
% \\&\multicolumn{1}{r}{\(\mid\)}&
%   \mtt{return} \opt{\mita{dot\_expr}} \mtt{;}
% \\&\multicolumn{1}{r}{\(\mid\)}&
%   \ttlb~\opt{\mita{statement\_sequence}} \ttrb
% \\&\multicolumn{1}{r}{\(\mid\)}&
%   NEST(\some{\mita{decl\_statement}} \(\mid\) \mita{expr}, \mita{stmt\_whencode})
% \end{tabular}}

  \RULE{\rt{single\_stmt}}
  \CASE{\NT{stmt}}
  \CASE{\NT{OR}(\NT{stmt})}

  \RULE{\rt{decl\_stmt}}
  \CASE{\mth{\T{metaid}^{\ssf{StmtList}}}}
  \CASE{\NT{decl\_var}}
  \CASE{\NT{stmt}}
  \CASE{\NT{OR}(\NT{stmt\_seq})}

  \RULE{\rt{stmt\_seq}}
  \CASE{\any{\NT{decl\_stmt}}
    \opt{\NT{DOTSEQ}(\some{\NT{decl\_stmt}},
      \NT{stmt\_whencode}) \any{\NT{decl\_stmt}}}}
  \CASE{\any{\NT{decl\_stmt}}
    \opt{\NT{DOTSEQ}(\NT{expr},
      \NT{stmt\_whencode}) \any{\NT{decl\_stmt}}}}

  \RULE{\rt{stmt\_whencode}}
  \CASE{when != \NT{XXXDOTSEQXXX}(\some{\NT{decl\_stmt}}, \NT{stmt\_whencode})}
  \CASE{when != \NT{XXXDOTSEQXXX}(\NT{expr}, \NT{stmt\_whencode})}

  \RULE{\rt{case\_line}}
  \CASE{default : \NT{fun\_start}}
  \CASE{case \NT{dot\_expr} : \NT{fun\_start}}

  \RULE{\rt{fun\_start}}
  \CASE{}

  \RULE{\rt{iter\_ident}}
  \CASE{\T{IteratorId}}
  \CASE{\T{MetaIterator}}
\end{grammar}

\noindent{\footnotesize\begin{tabular}{r@{\,\,\,}c@{\,\,\,}l}
% \mita{single\_statement} & ::= &
%      \mita{statement} \(\mid\) OR(\mita{statement})
% \\
% \mita{decl\_statement} & ::= &
%   \mth{\msf{metaid}^{\ssf{StmtList}}} \(\mid\)
%   \mita{decl\_var} \(\mid\) \mita{statement} \(\mid\)
%   OR(\mita{statement\_sequence})
% \\
% \mita{statement\_sequence} & ::= &\\
%   \multicolumn{3}{r}{\air\air\any{\mita{decl\_statement}}
%   \opt{DOTSEQ(\some{\mita{decl\_statement}} \(\mid\) \mita{expr},
%   \mita{stmt\_whencode}) \any{\mita{decl\_statement}}}}
% \\
\mita{stmt\_whencode} & ::= & \mtt{WHEN} \mtt{!=}
OPTDOTSEQ(\some{\mita{decl\_statement}} \(\mid\) \mita{expr},\mita{stmt\_whencode})
% \\
% OR(\mita{grammar}) & ::= &
%  \mtt{(} \mita{grammar} \ANY{\ttmid \mita{grammar}} \mtt{)}
\end{tabular}}

\begin{grammar}
  \RULE{\rt{OR}(\NT{grammar})}
  \CASE{( \NT{grammar} \ANY{\ttmid \NT{grammar}})}

  \RULE{\rt{DOTSEQ}(\NT{grammar},\NT{whencode})}
  \CASE{\ldots \opt{\NT{whencode}} \ANY{\NT{grammar} \ldots \opt{\NT{whencode}}}}

% \noindent{\footnotesize\begin{tabular}{r@{\,\,\,}c@{\,\,\,}l}
% DOTSEQ(\mita{grammar},\mita{whencode}) & ::= &
% \\ \multicolumn{3}{l}{\air\air\phantom{\(\mid\)}
%  \mtt{\ldots} \opt{\mita{whencode}} \ANY{\mita{grammar} \mtt{\ldots}
%  \opt{\mita{whencode}}}}
% \\ \multicolumn{3}{l}{\air\air\(\mid\)
%  \mtt{ooo} \opt{\mita{whencode}} \ANY{\mita{grammar} \mtt{ooo}
%  \opt{\mita{whencode}}}}
% \\ \multicolumn{3}{l}{\air\air\(\mid\)
%  \mtt{***} \opt{\mita{whencode}} \ANY{\mita{grammar} \mtt{***}
%  \opt{\mita{whencode}}}}
% \end{tabular}}

  \RULE{\rt{NEST}(\NT{grammar},\NT{whencode})}
  \CASE{<\ldots \NT{grammar} \ANY{\ldots \opt{\NT{whencode}} \NT{grammar}} \ldots>}
  \CASE{<+\ldots \NT{grammar} \ANY{\ldots \opt{\NT{whencode}} \NT{grammar}} \ldots+>}

% \noindent{\footnotesize\begin{tabular}{r@{\,\,\,}c@{\,\,\,}l}
% NEST(\mita{grammar},\mita{whencode}) & ::= &
% \\ \multicolumn{3}{l}{\air\air\phantom{\(\mid\)}
%  \mtt{<\ldots} \mita{grammar} \ANY{\ldots \opt{\mita{whencode}}
%  \mita{grammar}} \mtt{\ldots>}}
% \\ \multicolumn{3}{l}{\air\air\(\mid\)
%  \mtt{<ooo} \mita{grammar} \ANY{ooo \opt{\mita{whencode}} \mita{grammar}}
%  \mtt{ooo>}}
% \\ \multicolumn{3}{l}{\air\air\(\mid\)
%  \mtt{<***} \mita{grammar} \ANY{*** \opt{\mita{whencode}} \mita{grammar}}
%  \mtt{***>}}
% \end{tabular}}

\end{grammar}

%\vspace{\baselineskip}

\noindent
OR is a macro that generates a disjunction of patterns.  The three
tokens \T{(}, \T{\ttmid}, and \T{)} must appear in the leftmost
column, to differentiate them from the parentheses and bit-or tokens
that can appear within expressions (and cannot appear in the leftmost
column).  These tokens are furthermore different from (, \(\mid\), and
), which are part of the grammar metalanguage.

\section{Expressions}

A nest or a single ellipsis is allowed in some expression contexts, and
causes ambiguity in others.  For example, in a sequence \mtt{\ldots
\mita{expr} \ldots}, the nonterminal \mita{expr} must be instantiated as an
explicit C-language expression, while in an array reference,
\mtt{\mth{\mita{expr}_1} \mtt{[} \mth{\mita{expr}_2} \mtt{]}}, the
nonterminal \mth{\mita{expr}_2}, because it is delimited by brackets, can
be also instantiated as \mtt{\ldots}, representing an arbitrary expression.  To
distinguish between the various possibilities, we define three nonterminals
for expressions: {\em expr} does not allow either top-level nests or
ellipses, {\em nest\_expr} allows a nest but not an ellipsis, and {\em
dot\_expr} allows both.  The EXPR macro is used to express these variants
in a concise way.

%\vspace{\baselineskip}
\begin{grammar}
  \RULE{\rt{expr}}
  \CASE{\NT{EXPR}(\NT{expr})}

  \RULE{\rt{nest\_expr}}
  \CASE{\NT{EXPR}(\NT{nest\_expr})}
  \CASE{\NT{NEST}(\NT{nest\_expr}, \NT{exp\_whencode})}

  \RULE{\rt{dot\_expr}}
  \CASE{\NT{EXPR}(\NT{dot\_expr})}
  \CASE{\NT{NEST}(\NT{dot\_expr}, \NT{exp\_whencode})}
  \CASE{... \opt{\NT{exp\_whencode}}}

  \RULE{\rt{EXPR}(exp)}
  \CASE{\NT{exp} \NT{assign\_op} \NT{exp}}
  \CASE{\NT{exp}++}
  \CASE{\NT{exp}--}
  \CASE{\NT{unary\_op} \NT{exp}}
  \CASE{\NT{exp} \NT{bin\_op} \NT{exp}}
  \CASE{\NT{exp} ? \NT{dot\_expr} : \NT{exp}}
  \CASE{(\NT{type}) \NT{exp}}
  \CASE{\NT{exp} [\NT{dot\_expr}]}
  \CASE{\NT{exp} . \NT{id}}
  \CASE{\NT{exp} -> \NT{id}}
  \CASE{\NT{exp}(\opt{\NT{PARAMSEQ}(\NT{arg}, \NT{exp\_whencode})})}
  \CASE{\NT{id}}
  \CASE{\mth{\T{metaid}^{\ssf{Func}}}}
  \CASE{\mth{\T{metaid}^{\ssf{LocalFunc}}}}
  \CASE{\mth{\T{metaid}^{\ssf{Exp}}}}
  \CASE{\mth{\T{metaid}^{\ssf{Err}}}}
  \CASE{\mth{\T{metaid}^{\ssf{Const}}}}
  \CASE{\NT{const}}
  \CASE{(\NT{dot\_expr})}
  \CASE{\NT{OR}(\NT{exp})}

  \RULE{\rt{arg}}
  \CASE{\NT{nest\_expr}}
  \CASE{\mth{\T{metaid}^{\ssf{ExpList}}}}

  \RULE{\rt{exp\_whencode}}
  \CASE{when != \NT{dot\_expr}}

  \RULE{\rt{assign\_op}}
  \CASE{= \OR -= \OR += \OR *= \OR /= \OR \%=}
  \CASE{\&= \OR |= \OR \caret= \OR \lt\lt= \OR \gt\gt=}

  \RULE{\rt{bin\_op}}
  \CASE{* \OR / \OR \% \OR + \OR -}
  \CASE{\lt\lt \OR \gt\gt \OR \caret\xspace \OR \& \OR \ttmid}
  \CASE{< \OR > \OR <= \OR >= \OR == \OR != \OR \&\& \OR \ttmid\ttmid}

  \RULE{\rt{unary\_op}}
  \CASE{++ \OR -- \OR \& \OR * \OR + \OR - \OR !}

\end{grammar}

\section{Constant, Identifiers and Types for Transformations}

\begin{grammar}
  \RULE{const}
  \CASE{"\any{[\^{}"]}"}
  \CASE{[0-9]+}
  \CASE{\mth{\cdots}}

  \RULE{id}
  \CASE{\T{id} \OR \mth{\T{metaid}^{\ssf{Id}}}}

  \RULE{type}
  \CASE{\NT{ctype} \OR \mth{\T{metaid}^{\ssf{Type}}}}
\end{grammar}

% \noindent{\footnotesize\begin{tabular}{r@{\,\,\,}c@{\,\,\,}l}
% \mita{id} & ::= & \msf{id} \(\mid\) \mth{\msf{metaid}^{\ssf{Id}}}
% \end{tabular}}


\section{Examples}
%\label{sec:examples}

We provide some real or reallistic examples in the following section.
Each example is detailled with some C code on which its applied. The
description explains the matching process and when applicable the
rewriting rules.

\subsection{Function renaming}

One of the primary goal of Coccinelle is to perform software
evolution.  For instance, Coccinelle could be used to perform function
renaming. In the following example, every occurrence of function call
to \texttt{foo} is replaced by a call to the \texttt{bar} function.

\begin{tabular}{ccc}
Before & Semantic patch & After \\
\begin{minipage}[t]{.3\linewidth}
\begin{lstlisting}
#DEFINE TEST "foo"

printf("foo");

int main(int i) {
//Test
  int k = foo();

  if(1) {
    foo();
  } else {
    foo();
  }

  foo();
}
\end{lstlisting}
\end{minipage}
&
\begin{minipage}[t]{.3\linewidth}
\begin{lstlisting}[language=Cocci]
@M@@

@@@M


@-- foo()
@++ bar()
\end{lstlisting}
\end{minipage}
&
\begin{minipage}[t]{.3\linewidth}
\begin{lstlisting}
#DEFINE TEST "foo"

printf("foo");

int main(int i) {
//Test
  int k = bar();

  if(1) {
    bar();
  } else {
    bar();
  }

  bar();
}
\end{lstlisting}
\end{minipage}\\
\end{tabular}

\subsection{Removing a function argument}

Another important kind of evolution is the introduction or deletion of
function argument. In the following example, the \texttt{rule1} rule
looks for functions of type \texttt{irqreturn\_t} with two arguments. A
second anonymous rule, then looks for calls of the previously matched
functions which provides three arguments. The third argument is then
removed.

\begin{tabular}{c@{\hspace{1cm}}c}
\begin{lstlisting}[language=Cocci,name=arg]
@M@ rule1 @
identifier fn;
identifier irq, dev_id;
typedef irqreturn_t;
@@@M

static irqreturn_t fn (int irq, void *dev_id)
{
   ...
}

\end{lstlisting}
&
\begin{lstlisting}[language=Cocci,name=arg]
@M@@
identifier rule1.fn;
expression E1, E2, E3;
@@@M

 fn(E1, E2
@--  ,E3
   )
\end{lstlisting}\\
\end{tabular}

\begin{tabular}{c}
  \texttt{/drivers/atm/firestream.c} at line 1653 before\\
\begin{lstlisting}[language=PatchC]
static void fs_poll (unsigned long data)
{
        struct fs_dev *dev = (struct fs_dev *) data;

        fs_irq (0, dev, NULL);
        dev->timer.expires = jiffies + FS_POLL_FREQ;
        add_timer (&dev->timer);
}
\end{lstlisting}\\
\\
  \texttt{/drivers/atm/firestream.c} at line 1653 after\\
\begin{lstlisting}[language=PatchC]
static void fs_poll (unsigned long data)
{
        struct fs_dev *dev = (struct fs_dev *) data;

@-        fs_irq (0, dev);
        dev->timer.expires = jiffies + FS_POLL_FREQ;
        add_timer (&dev->timer);
}
\end{lstlisting}\\
\end{tabular}

\subsection{Introduction of a macro}

Finally, to avoid code duplication or error prone code, macros or
functions are introduced such as \texttt{BUG\_ON},
\texttt{DIV\_ROUND\_UP} ou \texttt{FIELD\_SIZE}. In these cases, the
semantic patch looks for the old code pattern and replaces it by the
new code.

For instance, the \texttt{BUG\_ON} macro enables to place some
assertion on expressions. However, the \texttt{BUG\_ON} macro could
not be used when the asserted expression may have some side-effects,
such as a function call, as a macro may evaluate more than once theirs
arguments.

\begin{tabular}{c@{\hspace{1cm}}c}
\begin{lstlisting}[language=Cocci,name=bugon]
@M@@
expression E,f;
@@@M

(
  if (<... f(...) ...>) { BUG(); }
|
@-- if (E) { BUG(); }
@++ BUG_ON(E);
)

\end{lstlisting}
&
\begin{lstlisting}[language=Cocci,name=bugon]
@M@ disable unlikely @
expression E,f;
@@@M

(
  if (<... f(...) ...>) { BUG(); }
|
@-- if (unlikely(E)) { BUG(); }
@++ BUG_ON(E);
)
\end{lstlisting}\\
\end{tabular}

\subsection{Look for \texttt{NULL} dereference}

This SmPL match looks for \texttt{NULL} dereferences. Once an
expression has been compared to \texttt{NULL}, a dereference to this
expression is prohibited unless the variable is reaffected.

\begin{tabular}{c@{\hspace{5mm}}p{.5\linewidth}}
Original & Semantic match \\
\begin{minipage}[t]{.5\linewidth}
\begin{lstlisting}
foo = kmalloc(1024);
if (!foo) {
  printk ("Error %s", foo->here);
  return;
}
foo->ok = 1;
return;
\end{lstlisting}
%
  \begin{center}
    Matched lines
  \end{center}

\begin{lstlisting}[language=Cocci]
foo = kmalloc(1024);
@-if (!foo) {
@-  printk ("Error %s", foo->here);
  return;
}
foo->ok = 1;
return;
\end{lstlisting}
\end{minipage}
&
\begin{minipage}[t]{.5\linewidth}
\begin{lstlisting}[language=Cocci]
@M@@
expression E, E1;
identifier f;
statement S1,S2,S3;
@@@M

@+* if (E == NULL)
{
  ... when != if (E == NULL) S1 else S2
      when != E = E1
@+* E->f
  ... when any
  return ...;
}
else S3
\end{lstlisting}
\end{minipage}
\end{tabular}

\subsection{Reference counter: the of\_xxx API}

Coccinelle could be used in conjunction with Python. Python code is
used inside special SmPL rule annoted with \texttt{script:python}.
Python rules inherit metavariables, such as identifier or token
positions, from other SmPL rules. The inherited metavariables could
then be manipulated by Python code.

The following semantic match looks for a call to the
\texttt{of\_find\_node\_by\_name} function. This call increment a
counter which must be decremented to release the resource. Then,
\texttt{when} there is no call to \texttt{of\_node\_put}, no new
affectation to the \texttt{device\_node} variable \texttt{n} and a
\texttt{return} statement is reached, a bug is detected and the
position \texttt{p1} and \texttt{p2} are initiliazed. As the python
only depends on these two position, it is evaluated. In the following
case, some emacs Org mode data are produced.

\begin{lstlisting}[language=Cocci]
@M@ r exists @
local idexpression struct device_node *n;
position p1, p2;
statement S1,S2;
expression E,E1;
@@@M

(
if (!(n@p1 = of_find_node_by_name(...))) S1
|
n@p1 = of_find_node_by_name(...)
)
<... when != of_node_put(n)
    when != if (...) { <+... of_node_put(n) ...+> }
    when != true !n  || ...
    when != n = E
    when != E = n
if (!n || ...) S2
...>
(
  return <+...n...+>;
|
return@p2 ...;
|
n = E1
|
E1 = n
)

@M@ script:python @
p1 << r.p1;
p2 << r.p2;
@@@M

print "* TODO [[view:%s::face=ovl-face1::linb=%s::colb=%s::cole=%s]
  [inc. counter:%s::%s]]"
    % (p1[0].file,p1[0].line,p1[0].column,p1[0].column_end,
      p1[0].file,p1[0].line)
print "[[view:%s::face=ovl-face2::linb=%s::colb=%s::cole=%s][return]]"
  % (p2[0].file,p2[0].line,p2[0].column,p2[0].column_end)
\end{lstlisting}

% \begin{tabular}{ccc}
% Before & Semantic patch & After \\
% \begin{minipage}[t]{.3\linewidth}
% \begin{lstlisting}
% \end{lstlisting}
% \end{minipage}
% &
% \begin{minipage}[t]{.3\linewidth}
% \begin{lstlisting}[language=Cocci]
% \end{lstlisting}
% \end{minipage}
% &
% \begin{minipage}[t]{.3\linewidth}
% \begin{lstlisting}
% \end{lstlisting}
% \end{minipage}\\
% \end{tabular}

%%% Local Variables:
%%% mode: LaTeX
%%% TeX-master: "cocci_syntax"
%%% coding: latin-9
%%% TeX-PDF-mode: t
%%% ispell-local-dictionary: "english"
%%% End:

\end{document}

%%% Local Variables:
%%% mode: LaTeX
%%% TeX-master: "cocci_syntax"
%%% coding: latin-9
%%% TeX-PDF-mode: t
%%% ispell-local-dictionary: "english"
%%% End:
